\chapter{Machine Learning}
\label{chap:machine_learning}


\newpage

The field of Machine Learning (ML) is concerned with algorithms which can be said to `learn' from experience, this may be contrasted with algorithms which achieve some task with a set of static instructions. The ability to learn allows ML algorithms to solve problems which may be too complex for a collection of explicity defined instructions. 
This chapter will give an overview of machine learning as relevant to the Higgs diphoton decay analysis at CMS and as a result certain definitions will not be as general. 

\section{Machine Learning Fundamentals}

A formal definition of learning is given in (ref): an algorithm is said to \textit{``learn from experience $E$ with respect to some class of tasks $T$ and performance measure $P$, if its performance at tasks in $T$, as measured by $P$, improves with experience $E$''}. 
The task $T$ may be defined in terms of the `features' of individual members of the dataset at what one is trying to achieve with them and these are typically represented as a vector $\vec{x}\in\mathbb{R}^{n}$ where each element of the vector is a feature of the datapoint. The two main classes of tasks we are interested in are classification and regression:
\begin{itemize}[leftmargin=.5in,noitemsep]
    \item Classification:
    \item Regression:
\end{itemize}

