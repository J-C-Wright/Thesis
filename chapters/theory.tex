\chapter{Theory}
\label{chap:theory}

\section{Introduction}
Modern particle theory is built upon the twin pillars of Yang-Mills theories and spontaneous symmetry breaking. 
Our best current model, the Standard Model (SM), is built from two such theories: electroweak theory and quantum chromodynamics, of which the former has its gauge symmetry spontaneously broken. 
In this chapter we will explore these two ideas before moving on to how they are used to construct the Standard Model with particular emphasis on the mechanism of symmetry breaking in the SM and its phenomenological consequences. 







\section{Yang-Mills Theories}
(Some introductory words)

\subsection{From Geometry to Gauge Fields}

The gauge covariant derivative $D_{\mu}$ and the field strength tensor $F_{\mu\nu}$ are two vital mathematical objects when one wants to construct the Lagrangian of a Yang-Mills theory.
Far from simply being an ansatz, they have a deep and elegant origin in the fundamental geometry of field theory. 
Their origin is outlined in this section in an unapologetically unrigourous manner: we start by describing the concept of a fibre bundle, its relationship to the internal symmetries of a field, and how the `warping' of a fibre bundle is related to the covariant derivative and the field strength tensor.


A fibre bundle $\mathcal{B}$ is a space which can be considered to consist of two parts: the base space $\mathcal{M}$ and the fibre $\mathcal{V}$. For each point $p$ in the base space there is an associated copy of the fibre space and these fibres do not intersect. 
In the context of a field one can consider these to be the external and internal spaces respectively. A special case is an ordinary product space where $\mathcal{B}$ is simply the Cartesian product of $\mathcal{M}$ and $\mathcal{V}$, generally one has more warped examples with curvature and less trivial topology. A visual example is given in figure X.
These warped examples are of interest in gauge theory, specifically when there is curvature in the fibre space with no torsion.


Particularly, we are interested in the cases when $\mathcal{V}$ is symmetric under some Lie group $\mathcal{G}$ which in our field of interest will be $\SUN$. 
These symmetries allow for the warping of the fibre bundle and correspond to the internal symmetries of a field. 
Furthermore, one can model these examples by taking the fibre to be $\mathcal{G}$ with the identity element not at a fixed location. 
The relevance of this will be apparent when we consider a fibre bundle section. A bundle section can be considered to `lift' the base space into the bundle: for each base space point we get a point within the associated fibre. In the gauge theory context choosing a section of the fibre bundle means choosing a particular $g(x)\in\mathcal{G}$, this is picking a gauge. 



To understand the warping of the fibre bundle we need the notion of a connection just like with the warped spaces of General Relativity.
This will allow for the introduction of warping to the internal space, and construction of invariants such as curvature and torsion tensors.
We can do this by constructing a differential operator $D_{\mu}$, and in our case of a fibre bundle with $\mathcal{V}=\mathcal{G}=\SUN$ and the internal space is simply stretched with no torsion we have,
\begin{equation}
\label{eq:theory:g_connection}
D_{\mu} = \partial_{\mu} - igA_{\mu}^{a} T^{a}.
\end{equation}
Where $A_{\mu}^{a}$ are generally complex-valued functions which depend on $x_{\mu}$ and operate by multiplying the input, and $T^{a}$ are the generators of the Lie group $\SUN$ which provide a basis in the fibre space with $a = 0,\ldots,N^{2}-1$.
We recognise this as having the familiar form of the gauge covariant derivative and the $A_{\mu}^{a}$ as the gauge potential. 


Now we have the connection we can begin to construct invariants of the geometry of the internal space. In particular we can construct the curvature tensor as follows
\begin{equation}
    \label{eq:theory:g_curvature}
    \frac{i}{g}[D_{\mu},D_{\nu}] = F_{\mu\nu} = \partial_{\mu} A^{b}_{\nu} T^{b} - \partial_{\nu} A^{a}_{\mu} T^{a} - ig[A^{a}_{\mu}, A^{b}_{\nu}].
\end{equation}
We recognise this form as the field strength tensor.

One can now see what is happening when a global symmetry is promoted to a gauge symmetry: we have induced some non-trivial warping of the field's internal space which gives rise to the $A^{a}_{\mu}$ gauge fields and their kinematics through the curvature $F_{\mu\nu}$.




\subsection{Constructing a Lagrangian}
With these ingredients we can construct a generic Yang-Mills Lagrangian with a straightforward procedure: we begin with a global symmetry of the fields which we promote to a gauge symmetry, we construct the gauge covariant derivative, replace $\partial_{\mu} \rightarrow D_{\mu}$ in the free theory, and add an interaction term based on the field strength tensor. 
Let us consider, as a concrete example, the collection of massive free Dirac fermions which we will turn into an interacting gauge theory with $\mathcal{G} = \SUN$.
We first construct the gauge covariant derivative, 
\begin{equation}
    \label{eq:theory:generic_SUN_Dmu}
    D_{\mu} = \partial_{\mu} - igA_{\mu}^{a} T^{a},
\end{equation}
and replace $\partial_{\mu} \rightarrow D_{\mu}$ in the free Lagrangian
\begin{equation}
    \label{eq:theory:int_dirac_no_gauge_dynamics}
    \mathcal{L} = \sum_{\alpha} \bar{\Psi}^{\alpha}[i\gamma^{\mu}(D_{\mu}\Psi)^{\alpha} - m\Psi^{\alpha}].
\end{equation}
We must also introduce a kinematic term for the gauge fields, but the contraction of the general non-abelian field strength tensor with itself is not gauge invariant, only its trace over the generator indices is. 
We us this as the gauge-invariant kinetic for our final Yang-Mills Lagranian,
\begin{equation}
    \label{eq:theory:int_dirac_gauge_dynamics}
    \mathcal{L} = \sum_{\alpha} \bar{\Psi}^{\alpha}[i\gamma^{\mu}(D_{\mu}\Psi)^{\alpha} - m\Psi^{\alpha}] - \frac{1}{2}\mathrm{Tr}F_{\mu\nu}F^{\mu\nu}.
\end{equation}






%Generic description of gauging a symmetry, constructing a covariant derivative, getting a gauge field mediating the interaction controlled by gauge coupling, dynamics of the gauge field from the field strength tensor, how to construct a Lagrangian. An aside on renormalisability of YM theories?





\subsection{Phenomenology}
To analyse what sort of particle interactions occur in this theory let us 'unpack' equation \ref{eq:theory:int_dirac_gauge_dynamics} and isolate the interaction terms.
Firstly, the gauge covariant derivative introduces a trilinear coupling between a gauge field and two fermionswhich is proportional to $g$


 \feynmandiagram [horizontal=f2 to f3] {
     f1 -- [fermion] f2 -- [fermion] f3 -- [fermion] f4, f2 -- [photon] p1,
     f3 -- [photon] p2,
 };






\section{Spontaneous Symmetry Breaking}

Solution doesn't obey the symmetry of the theory etc. Example of the ferromagnet with a nice diagram. Requires QFT not just QM. 




\subsection{Gauge Symmetry Breaking}
The above in the context of a gauge theory. Talk about U(1) and superconductors. Goldstones from global symmetries, no goldstones in gauge theory, massive bosons, higgs boson. Higgs potential diagram. 



\section{The Standard Model of Particle Physics}
More introduction to the SM. Its phenomenological motivation. Its sucesses and failures. 

\subsection{The problems with fundamental mass}
Renormalisation and gauge invariance issues, need Higgs mechanism for heavy bosons, and Yukawas for fermionic masses.

\subsection{The electroweak sector and the Higgs field}
Build the Lagrangian with reference to the above. Introduce the Higgs sector and how it breaks the SU(2) gauge subgroup.

\subsection{Quantum Chromodynamics}
Build the QCD lagrangian, describing quarks and gluons, flavour and colour, why confinement, confinement and hadrons and jets, non-exact symmetries of QCD and pions and stuff?

\subsection{The Higgs boson: couplings, production and decay}
The Higgs itself, what it couples to and why, how it is produced at the LHC, how it decays. Emphasis on the VBF production mode and the diphoton decay mode. 
